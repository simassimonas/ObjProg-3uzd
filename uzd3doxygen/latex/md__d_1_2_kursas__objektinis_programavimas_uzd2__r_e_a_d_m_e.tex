\subsection*{Programos paleidimas naudojant {\ttfamily G\+NU C++ Compiler}}

{\bfseries{Tiesiog per terminala}}
\begin{DoxyItemize}
\item {\ttfamily g++ -\/o run main.\+cpp}
\item {\ttfamily ./run}
\end{DoxyItemize}

{\bfseries{Su Makefile}}

Programos sudarymas\+:
\begin{DoxyItemize}
\item {\ttfamily make}
\end{DoxyItemize}

Sudarytu failu istrynimas\+:
\begin{DoxyItemize}
\item {\ttfamily clean}
\end{DoxyItemize}

{\bfseries{Po v0.\+3}}

Programos sudarymas\+:
\begin{DoxyItemize}
\item {\ttfamily make}
\end{DoxyItemize}

functions.\+cpp failo sukompiliavimas\+:
\begin{DoxyItemize}
\item {\ttfamily fun}
\end{DoxyItemize}

Sudarytu failu istrynimas\+:
\begin{DoxyItemize}
\item {\ttfamily clean}
\end{DoxyItemize}

\subsection*{Versiju istorija}

\subsubsection*{\href{https://github.com/simassimonas/ObjProg-2uzd/releases/tag/v0.1}{\texttt{ v0.\+1}}}

{\bfseries{Pridėta}}
\begin{DoxyItemize}
\item Pirmine programos versija
\item .gitignore failas
\item Makefile
\end{DoxyItemize}

{\bfseries{Pastebejimas}}

Esant tokiam kodui, ciklas veiks, kol ivedami int\textquotesingle{}ai ir tarkim ivedus raide, jis sustos ir nenuskaitys reiksmes, taciau ivedus float\textquotesingle{}a, pvz., 2.\+3 ciklas sustos, bet pries tai nuskaitys reiksme, ja suapvalins iki 2 ir isves i ekrana


\begin{DoxyCode}{0}
\DoxyCodeLine{int x;}
\DoxyCodeLine{    while(cin >> x) \{}
\DoxyCodeLine{    cout << x << endl;}
\DoxyCodeLine{    \}}
\end{DoxyCode}


\subsubsection*{\href{https://github.com/simassimonas/ObjProg-2uzd/releases/tag/v0.1.1}{\texttt{ v0.\+1.\+1}}}

{\bfseries{Pridėta}}
\begin{DoxyItemize}
\item cmasyvas.\+cpp file\textquotesingle{}as, kuriame n.\+d rezultatai saugomi ne vectoriuje, o masyve
\end{DoxyItemize}

\subsubsection*{\href{https://github.com/simassimonas/ObjProg-2uzd/releases/tag/v0.2}{\texttt{ v0.\+2}}}

{\bfseries{Pridėta}}
\begin{DoxyItemize}
\item Antrine programos versija
\item duom.\+txt failas
\end{DoxyItemize}

Pirmas skaicius nurodo, kiek bus studentu, antras skaicius nurodo, kiek bus n.\+d pazymiu, o kiekvienos eilutes paskutinis skaitmuo yra egzamino rezultatas


\begin{DoxyCode}{0}
\DoxyCodeLine{5 3}
\DoxyCodeLine{vardas1 vardas2 3 6 8 3}
\DoxyCodeLine{vardas2 vardas2 5 7 3 1}
\DoxyCodeLine{vardas3 vardas2 1 6 10 6}
\DoxyCodeLine{vardas4 vardas4 6 4 1 10}
\DoxyCodeLine{vardas5 vardas5 7 4 2 7}
\end{DoxyCode}


{\bfseries{Papildyta}}
\begin{DoxyItemize}
\item R\+E\+A\+D\+M\+E.\+md failas
\end{DoxyItemize}

{\bfseries{Pastebejimas}}

Deja, laiku nespejau padaryti vardu rusiavimo ir parasyti tvarkingu komentaru, bet tai padarysiu su kitu commitu arba blogiausiu atveju kitu releas\textquotesingle{}u.

\subsubsection*{\href{https://github.com/simassimonas/ObjProg-2uzd/releases/tag/v0.2.1}{\texttt{ v0.\+2.\+1}}}

{\bfseries{Pridėta}}
\begin{DoxyItemize}
\item Pridėtas studentų rūšiavimas pagal vardus
\end{DoxyItemize}

{\bfseries{Papildyta}}
\begin{DoxyItemize}
\item Patobulintas nuskaitymas is failo
\end{DoxyItemize}

Nebereikia pirmu dvieju skaiciu, nurodanciu, kiek bus studentu ir kiek n.\+d pazymiu jie tures. Dabar kiekvienas studentas gali tureti skirtinga n.\+d pazymiu kieki, o paskutinis skaicius nurodo egzamino rezultata


\begin{DoxyCode}{0}
\DoxyCodeLine{vardas4 vardas 3 6 8 3 6 }
\DoxyCodeLine{vardas3 vardas 5 7 1 4}
\DoxyCodeLine{vardas1 vardas 1 6 10 6 2 6 4 }
\DoxyCodeLine{vardas5 vardas 6 4 5}
\DoxyCodeLine{vardas2 vardas 7 4 2 7}
\end{DoxyCode}
 \subsubsection*{\href{https://github.com/simassimonas/ObjProg-2uzd/releases/tag/v0.3}{\texttt{ v0.\+3}}}

{\bfseries{Pridėta}}
\begin{DoxyItemize}
\item exception handling
\item header ir funkciju failai
\end{DoxyItemize}

{\bfseries{Papildyta}}
\begin{DoxyItemize}
\item updated makefile
\end{DoxyItemize}

\subsubsection*{\href{https://github.com/simassimonas/ObjProg-2uzd/releases/tag/v0.4}{\texttt{ v0.\+4}}}

{\bfseries{Pridėta}}
\begin{DoxyItemize}
\item Ketvirta programos versija
\item galimybe generuoti atsitiktinius studentu sarasu failus
\item studentu rusiavimas i dvi grupes ir isvedimas i skirtingus failus (vargsiukai ir kietakai)
\item programos veikimo laiko matavimas
\end{DoxyItemize}

{\bfseries{Programos veikimo laiko testavimas generuojant skirtingo dydzio duomenu failus}}

\tabulinesep=1mm
\begin{longtabu}spread 0pt [c]{*{2}{|X[-1]}|}
\hline
\PBS\centering \cellcolor{\tableheadbgcolor}\textbf{ Irašų sk.  }&\PBS\centering \cellcolor{\tableheadbgcolor}\textbf{ Trukmė   }\\\cline{1-2}
\endfirsthead
\hline
\endfoot
\hline
\PBS\centering \cellcolor{\tableheadbgcolor}\textbf{ Irašų sk.  }&\PBS\centering \cellcolor{\tableheadbgcolor}\textbf{ Trukmė   }\\\cline{1-2}
\endhead
10  &3ms   \\\cline{1-2}
100  &10ms   \\\cline{1-2}
1000  &42ms   \\\cline{1-2}
10000  &308ms   \\\cline{1-2}
100000  &2930ms   \\\cline{1-2}
1000000  &30431ms   \\\cline{1-2}
\end{longtabu}


\subsubsection*{\href{https://github.com/simassimonas/ObjProg-2uzd/releases/tag/v0.5}{\texttt{ v0.\+5}}}

{\bfseries{Pridėta}}
\begin{DoxyItemize}
\item pridetos persidengiancios funkcijos, todel skaiciavimus galima atlikti su vector, deque ir list tipo konteineriais
\end{DoxyItemize}

{\bfseries{Programos veikimo laiko testavimas naudojant vienodo dydzio duomenu failus, bet skirtingo tipo konteinerius}}

\tabulinesep=1mm
\begin{longtabu}spread 0pt [c]{*{3}{|X[-1]}|}
\hline
\PBS\centering \cellcolor{\tableheadbgcolor}\textbf{ Konteineris  }&\PBS\centering \cellcolor{\tableheadbgcolor}\textbf{ 100000irašų  }&\PBS\centering \cellcolor{\tableheadbgcolor}\textbf{ 500000irašų   }\\\cline{1-3}
\endfirsthead
\hline
\endfoot
\hline
\PBS\centering \cellcolor{\tableheadbgcolor}\textbf{ Konteineris  }&\PBS\centering \cellcolor{\tableheadbgcolor}\textbf{ 100000irašų  }&\PBS\centering \cellcolor{\tableheadbgcolor}\textbf{ 500000irašų   }\\\cline{1-3}
\endhead
\PBS\centering Vector  &\PBS\centering 2181ms  &\PBS\centering 10901ms   \\\cline{1-3}
\PBS\centering Deque  &\PBS\centering 2267ms  &\PBS\centering 11592ms   \\\cline{1-3}
\PBS\centering List  &\PBS\centering 121340ms  &\PBS\centering error   \\\cline{1-3}
\end{longtabu}


{\ttfamily sitoj versijoje blogai padaryta su list konteineriu, v0.\+5.\+1 pataisyta}

\subsubsection*{\href{https://github.com/simassimonas/ObjProg-2uzd/releases/tag/v0.5.1}{\texttt{ v0.\+5.\+1}}}

{\bfseries{Papildyta}}
\begin{DoxyItemize}
\item buvau pridares nesamoniu su list konteineriu, todel skaiciavimai truko neadekvaciai ilgai, dabar viskas sutvarkyta
\end{DoxyItemize}

\tabulinesep=1mm
\begin{longtabu}spread 0pt [c]{*{3}{|X[-1]}|}
\hline
\PBS\centering \cellcolor{\tableheadbgcolor}\textbf{ Konteineris  }&\PBS\centering \cellcolor{\tableheadbgcolor}\textbf{ 100000irašų  }&\PBS\centering \cellcolor{\tableheadbgcolor}\textbf{ 500000irašų   }\\\cline{1-3}
\endfirsthead
\hline
\endfoot
\hline
\PBS\centering \cellcolor{\tableheadbgcolor}\textbf{ Konteineris  }&\PBS\centering \cellcolor{\tableheadbgcolor}\textbf{ 100000irašų  }&\PBS\centering \cellcolor{\tableheadbgcolor}\textbf{ 500000irašų   }\\\cline{1-3}
\endhead
\PBS\centering List  &\PBS\centering 2136ms  &\PBS\centering 10730ms   \\\cline{1-3}
\end{longtabu}


\subsubsection*{\href{https://github.com/simassimonas/ObjProg-2uzd/releases/tag/v1.0}{\texttt{ v1.\+0}}}

{\bfseries{Pridėta}}
\begin{DoxyItemize}
\item antra startegija
\item skaiciavimas su std\+::partition algoritmu
\end{DoxyItemize}

{\bfseries{Programos veikimo laiko testavimas naudojant skirtingo dydzio duomenu failus ir skirtingas strategijas}}

{\bfseries{Vector}}

\tabulinesep=1mm
\begin{longtabu}spread 0pt [c]{*{3}{|X[-1]}|}
\hline
\PBS\centering \cellcolor{\tableheadbgcolor}\textbf{ Irašų sk.  }&\PBS\centering \cellcolor{\tableheadbgcolor}\textbf{ 1 strategija  }&\PBS\centering \cellcolor{\tableheadbgcolor}\textbf{ 2 strategija   }\\\cline{1-3}
\endfirsthead
\hline
\endfoot
\hline
\PBS\centering \cellcolor{\tableheadbgcolor}\textbf{ Irašų sk.  }&\PBS\centering \cellcolor{\tableheadbgcolor}\textbf{ 1 strategija  }&\PBS\centering \cellcolor{\tableheadbgcolor}\textbf{ 2 strategija   }\\\cline{1-3}
\endhead
\PBS\centering 1000  &\PBS\centering 26ms  &\PBS\centering 128ms   \\\cline{1-3}
\PBS\centering 10000  &\PBS\centering 240ms  &\PBS\centering 9s   \\\cline{1-3}
\PBS\centering 100000  &\PBS\centering 2,2s  &\PBS\centering 897s   \\\cline{1-3}
\PBS\centering 1000000  &\PBS\centering 22s  &\PBS\centering Per ilgai   \\\cline{1-3}
\end{longtabu}


{\bfseries{Deque}}

\tabulinesep=1mm
\begin{longtabu}spread 0pt [c]{*{3}{|X[-1]}|}
\hline
\PBS\centering \cellcolor{\tableheadbgcolor}\textbf{ Irašų sk.  }&\PBS\centering \cellcolor{\tableheadbgcolor}\textbf{ 1 strategija  }&\PBS\centering \cellcolor{\tableheadbgcolor}\textbf{ 2 strategija   }\\\cline{1-3}
\endfirsthead
\hline
\endfoot
\hline
\PBS\centering \cellcolor{\tableheadbgcolor}\textbf{ Irašų sk.  }&\PBS\centering \cellcolor{\tableheadbgcolor}\textbf{ 1 strategija  }&\PBS\centering \cellcolor{\tableheadbgcolor}\textbf{ 2 strategija   }\\\cline{1-3}
\endhead
\PBS\centering 1000  &\PBS\centering 24ms  &\PBS\centering 70ms   \\\cline{1-3}
\PBS\centering 10000  &\PBS\centering 240ms  &\PBS\centering 3.\+5s   \\\cline{1-3}
\PBS\centering 100000  &\PBS\centering 2,2s  &\PBS\centering 375s   \\\cline{1-3}
\PBS\centering 1000000  &\PBS\centering 23s  &\PBS\centering Per ilgai   \\\cline{1-3}
\end{longtabu}


{\bfseries{List}}

\tabulinesep=1mm
\begin{longtabu}spread 0pt [c]{*{3}{|X[-1]}|}
\hline
\PBS\centering \cellcolor{\tableheadbgcolor}\textbf{ Irašų sk.  }&\PBS\centering \cellcolor{\tableheadbgcolor}\textbf{ 1 strategija  }&\PBS\centering \cellcolor{\tableheadbgcolor}\textbf{ 2 strategija   }\\\cline{1-3}
\endfirsthead
\hline
\endfoot
\hline
\PBS\centering \cellcolor{\tableheadbgcolor}\textbf{ Irašų sk.  }&\PBS\centering \cellcolor{\tableheadbgcolor}\textbf{ 1 strategija  }&\PBS\centering \cellcolor{\tableheadbgcolor}\textbf{ 2 strategija   }\\\cline{1-3}
\endhead
\PBS\centering 1000  &\PBS\centering 30ms  &\PBS\centering 30ms   \\\cline{1-3}
\PBS\centering 10000  &\PBS\centering 270ms  &\PBS\centering 212ms   \\\cline{1-3}
\PBS\centering 100000  &\PBS\centering 2,1s  &\PBS\centering 2.\+1s   \\\cline{1-3}
\PBS\centering 1000000  &\PBS\centering 22s  &\PBS\centering 21s   \\\cline{1-3}
\end{longtabu}


{\bfseries{Skaiciavimu trukme, pritaikius std\+::partition algoritma vectoriui}}

\tabulinesep=1mm
\begin{longtabu}spread 0pt [c]{*{2}{|X[-1]}|}
\hline
\PBS\centering \cellcolor{\tableheadbgcolor}\textbf{ Irašų sk.  }&\PBS\centering \cellcolor{\tableheadbgcolor}\textbf{ std\+::partition   }\\\cline{1-2}
\endfirsthead
\hline
\endfoot
\hline
\PBS\centering \cellcolor{\tableheadbgcolor}\textbf{ Irašų sk.  }&\PBS\centering \cellcolor{\tableheadbgcolor}\textbf{ std\+::partition   }\\\cline{1-2}
\endhead
\PBS\centering 1000  &\PBS\centering 26ms   \\\cline{1-2}
\PBS\centering 10000  &\PBS\centering 230ms   \\\cline{1-2}
\PBS\centering 100000  &\PBS\centering 2,1s   \\\cline{1-2}
\PBS\centering 1000000  &\PBS\centering 21s   \\\cline{1-2}
\end{longtabu}


\subsubsection*{\href{https://github.com/simassimonas/ObjProg-2uzd/releases/tag/v1.1}{\texttt{ P\+A\+P\+I\+L\+D\+O\+MA U\+Z\+D\+U\+O\+T\+IS}}}

{\bfseries{Pridėta}}
\begin{DoxyItemize}
\item rask\+Minkstus() ir iterpk\+Kietus() funkcijos
\end{DoxyItemize}

\tabulinesep=1mm
\begin{longtabu}spread 0pt [c]{*{4}{|X[-1]}|}
\hline
\PBS\centering \cellcolor{\tableheadbgcolor}\textbf{ Irašų sk.  }&\PBS\centering \cellcolor{\tableheadbgcolor}\textbf{ rask\+Minkstus()  }&\PBS\centering \cellcolor{\tableheadbgcolor}\textbf{ iterpk\+Kietus(vector)  }&\PBS\centering \cellcolor{\tableheadbgcolor}\textbf{ terpk\+Kietus(deque)   }\\\cline{1-4}
\endfirsthead
\hline
\endfoot
\hline
\PBS\centering \cellcolor{\tableheadbgcolor}\textbf{ Irašų sk.  }&\PBS\centering \cellcolor{\tableheadbgcolor}\textbf{ rask\+Minkstus()  }&\PBS\centering \cellcolor{\tableheadbgcolor}\textbf{ iterpk\+Kietus(vector)  }&\PBS\centering \cellcolor{\tableheadbgcolor}\textbf{ terpk\+Kietus(deque)   }\\\cline{1-4}
\endhead
\PBS\centering 10000  &\PBS\centering 9.\+2s  &\PBS\centering 9ms  &\PBS\centering 13ms   \\\cline{1-4}
\PBS\centering 100000  &\PBS\centering 894s  &\PBS\centering 90ms  &\PBS\centering 150ms   \\\cline{1-4}
\end{longtabu}


\subsubsection*{\href{https://github.com/simassimonas/ObjProg-3uzd/releases/tag/v1.1}{\texttt{ v1.\+1}}}

{\bfseries{Pridėta}}
\begin{DoxyItemize}
\item studento klase
\end{DoxyItemize}

{\bfseries{Skaiciavimu trukme, pritaikius std\+::partition algoritma ir vectoriu naudojant sena struct}}

\tabulinesep=1mm
\begin{longtabu}spread 0pt [c]{*{2}{|X[-1]}|}
\hline
\PBS\centering \cellcolor{\tableheadbgcolor}\textbf{ Irašų sk.  }&\PBS\centering \cellcolor{\tableheadbgcolor}\textbf{ std\+::partition   }\\\cline{1-2}
\endfirsthead
\hline
\endfoot
\hline
\PBS\centering \cellcolor{\tableheadbgcolor}\textbf{ Irašų sk.  }&\PBS\centering \cellcolor{\tableheadbgcolor}\textbf{ std\+::partition   }\\\cline{1-2}
\endhead
\PBS\centering 10000  &\PBS\centering 230ms   \\\cline{1-2}
\PBS\centering 100000  &\PBS\centering 2,1s   \\\cline{1-2}
\PBS\centering 500000  &\PBS\centering 21s   \\\cline{1-2}
\end{longtabu}


{\bfseries{Skaiciavimu trukme, pritaikius std\+::partition algoritma ir vectoriu naudojant nauja class}}

\tabulinesep=1mm
\begin{longtabu}spread 0pt [c]{*{2}{|X[-1]}|}
\hline
\PBS\centering \cellcolor{\tableheadbgcolor}\textbf{ Irašų sk.  }&\PBS\centering \cellcolor{\tableheadbgcolor}\textbf{ std\+::partition   }\\\cline{1-2}
\endfirsthead
\hline
\endfoot
\hline
\PBS\centering \cellcolor{\tableheadbgcolor}\textbf{ Irašų sk.  }&\PBS\centering \cellcolor{\tableheadbgcolor}\textbf{ std\+::partition   }\\\cline{1-2}
\endhead
\PBS\centering 10000  &\PBS\centering 430ms   \\\cline{1-2}
\PBS\centering 100000  &\PBS\centering 4,3s   \\\cline{1-2}
\PBS\centering 500000  &\PBS\centering 21s   \\\cline{1-2}
\end{longtabu}


{\bfseries{Skaiciavimu trukme, pritaikius std\+::partition algoritma, vectoriu ir naudojant nauja class bei optimization flags (sie skaiciavimai buvo atlikti ne ant Windows\textquotesingle{}u, o ant Ubuntu, todel greiciai gerokai skiriasi nuo praeitu)}}

\tabulinesep=1mm
\begin{longtabu}spread 0pt [c]{*{5}{|X[-1]}|}
\hline
\PBS\centering \cellcolor{\tableheadbgcolor}\textbf{ Irašų sk.  }&\PBS\centering \cellcolor{\tableheadbgcolor}\textbf{ default  }&\PBS\centering \cellcolor{\tableheadbgcolor}\textbf{ O1  }&\PBS\centering \cellcolor{\tableheadbgcolor}\textbf{ O2  }&\PBS\centering \cellcolor{\tableheadbgcolor}\textbf{ O3   }\\\cline{1-5}
\endfirsthead
\hline
\endfoot
\hline
\PBS\centering \cellcolor{\tableheadbgcolor}\textbf{ Irašų sk.  }&\PBS\centering \cellcolor{\tableheadbgcolor}\textbf{ default  }&\PBS\centering \cellcolor{\tableheadbgcolor}\textbf{ O1  }&\PBS\centering \cellcolor{\tableheadbgcolor}\textbf{ O2  }&\PBS\centering \cellcolor{\tableheadbgcolor}\textbf{ O3   }\\\cline{1-5}
\endhead
\PBS\centering 10000  &\PBS\centering 100ms  &\PBS\centering 77ms  &\PBS\centering 73ms  &\PBS\centering 71ms   \\\cline{1-5}
\PBS\centering 100000  &\PBS\centering 1050Ms  &\PBS\centering 750ms  &\PBS\centering 750ms  &\PBS\centering 750ms   \\\cline{1-5}
\PBS\centering 500000  &\PBS\centering 5s  &\PBS\centering 3,7s  &\PBS\centering 3,7s  &\PBS\centering 3,7s   \\\cline{1-5}
\end{longtabu}
